\documentclass[main.tex]{subfiles}
\begin{document}

\section{The axioms of time evolution}

Let us consider the axioms of quantum mechanical time evolution:

\begin{enumerate}
    \item States are described by kets at some starting time \(t_0\): \(\ket{\alpha} = \ket{\alpha, t_0}\);
    \item their time evolution is described by a linear operator \(U(t; t_0)\) as \(\ket{\alpha, t_0 ; t} = U(t, t_0) \ket{\alpha}\);
    \item this time evolution obeys \(\lim_{t \rightarrow t_0} \ket{\alpha, t;t_0}  = \ket{\alpha; t_0} \). \label{item:identity-property}
\end{enumerate}

If we have an observable \(A\), we can express our states with respect to its  eigenbasis \(\ket{\alpha'} \):
%
\begin{equation}
  \ket{\alpha, t_0} = \sum _{\alpha'} c_{\alpha'} (t_0) \ket{\alpha'}\,,
\end{equation}
 %
and do the same for their evolved counterparts, with evolved coefficients \(c_{\alpha'}(t)\).
If \([A, H]=0\) then \(\abs{c_{\alpha'}(t)} = \abs{c_{\alpha'}(t_0)} \), while in general this does not hold.

We impose the normalization of kets: \(1 = \braket{\alpha}{\alpha} = \sum _{\alpha'} \abs{c_{\alpha'}(t)}^2 \) for all times. This directly implies that the linear operator \(U\) must be unitary: \(U^\dag U = \id\).

Also, we impose the composition law: if \(t_2 \geq t_1 \geq t_0\), then \(U(t_2;t_0) \overset{!}{=} U(t_2;t_1) U(t_1;t_0)\).

Condition \ref{item:identity-property} means that we can expand:
%
\begin{equation}
  U(t_0 + \dd{t}, t_0) = \id + \frac{H \dd{t}}{i \hbar}
\end{equation}
%
with some self-adjoint operator \(H\).
Manipulating this, for a generic time \(t\) we get:
%
\begin{equation}
  \dv{}{t} U(t;t_0) = \frac{HU}{i \hbar} \,,
\end{equation}
%
which directly implies the Schrödinger equation
%
\begin{equation}
  i \hbar \partial_t \ket{\alpha, t_0; t} = H \ket{\alpha, t_0; t}
\end{equation}

\subsection{Time-independent Hamiltonians}

We can Taylor-expand the evolution operator as such:

\begin{subequations}
\begin{align}
  U(t, t_0) &= \exp(\frac{H (t-t_0)}{i \hbar} )  \\
  &= \id + \frac{H (t-t_0)}{i \hbar} + \frac{1}{2} \frac{H^2 (t-t_0)^2}{(i \hbar^2)^2} + o(\abs{t-t_0}^2 )\,,
\end{align}
\end{subequations}
%
and also its derivative:
%
\begin{subequations}
\begin{align}
  \partial_t U(t, t_0) &= \frac{H}{i \hbar} + \frac{1}{2} \frac{H^2}{(i \hbar)^2} 2t + O(\abs{t-t_0}^2)   \\
  &= \frac{HU}{i \hbar} \,,
\end{align}
\end{subequations}
%
which is consistent with the Schrödinger equation.

Alternatively, we can look at the limit of infinitesimal time evolutions:

\begin{equation}
  \qty(\id + \frac{H}{i \hbar} \frac{t}{N})^N \xrightarrow{N\rightarrow\infty}
 \exp( \frac{Ht}{i \hbar} )
\end{equation}

In general, continuous symmetries have representations in a Hilbert space: say we have the time translation symmetry \(P(\xi) t = t + \xi\); its representation is generally in the form \(\exp(-i \Omega t) \) for a self-adjoint \(\Omega\).

The fact that the generator of the time-translation symmetry is the Hamiltonian: \(\Omega = H / \hbar\) is a postulate.

\subsubsection{Time-dependent commuting Hamiltonians}

We assume that \(H = H(t)\), but \([H(t_1), H(t_2)] = 0\) for any \(t_{1, 2}\).
Then:
%
\begin{equation}
  U(t, t_0) = \exp(\frac{1}{i \hbar} \int_{t_0}^t  H(\tau) \dd{\tau}) \,,
\end{equation}
%
which can be proved like before, substituting \(Ht \rightarrow \int^t H(\tau) \dd{\tau}\).

\subsubsection{Time-dependent non-commuting Hamiltonians}

In general, if \([H(t_1), H(t_2)] \neq 0\), we have the following expression by Dyson:
%
\begin{equation}
  U(t, t_0) = \id + \sum_N^{\infty}
  \frac{1}{(i \hbar)^N} \int_{t_0}^{t}\dd{t_1} \int_{t_0}^{t_1}\dd{t_2} \dots
  \int_{t_0}^{t_{n-1}} \dd{t_n} \pi_i H (t_i)\,.
\end{equation}

A concrete Hamiltonian for which we'd need this is something in the form \(H = - \mu \cdot B(t)\), where the magnetic field \(B(t)\) changes directions.
Since \(\mu \cdot B \propto s \cdot B\) and the \(s_j\) do not commute, this does not commute with itself at different times.  

\end{document}
