\documentclass[main.tex]{subfiles}
\begin{document}

\section{Before the PI}

\subsection{Schrödinger, Heisenberg \& interaction}

We denote \(U = \exp(Ht/i \hbar) \), and similarly with \(H_0 \rightarrow U_0 \), \(V \rightarrow U_V\).

\paragraph{Schrödinger}

\begin{enumerate}
    \item State kets are \(\ket{\psi (t)} = U \ket{\psi (t = 0)}  \);
    \item observables are \(A(t) \equiv A(t=0)\);
    \item base kets are defined by \(A \ket{a} = a \ket{a} \), therefore \(\ket{a(t)} \equiv \ket{a(t=0)} \).
\end{enumerate}

\paragraph{Heisenberg}

\begin{enumerate}
    \item State kets are \(\ket{\psi (t)} \equiv \ket{\psi (t=0)} \);
    \item observables are \(A(t) = U ^\dag A(t=0) U \);
    \item base kets are \(\ket{a(t)} = U ^\dag \ket{a(t=0)} \).
\end{enumerate}
    

\paragraph{Interaction}

We denote by a subscript \(S\) or \(I\) objects in the Schrödinger or interaction system. In the 

\begin{enumerate}
    \item State kets are defined as \(\ket{\psi (t)}_I = U_0 ^\dag \ket{\psi (t)}_S \);
    \item observables are defined as \(A_I (t) = U_0 ^\dag A_S U_0 \);
    \item as base kets we use eigenstates of \(H_0 \): \(H_0 \ket{n} = E_n \ket{n} \). These evolve like \(\ket{n(t)} = U_0 \ket{n(t= 0)} \).
\end{enumerate}

Then, we can generically write the evolution of a Schrödinger ket as 
%
\begin{equation}
  \ket{\psi (t)}_S = 
  \sum _{n}  c_n(t) \exp(E_n t / i \hbar) \ket{n}
\,,
\end{equation}
%
therefore the evolution of the interaction ket is 
%
\begin{equation}
    \ket{\psi (t)}_I = 
    U_0 ^\dag \ket{\psi (t)} _S = \sum _n c_n (t) \ket{n} 
\,.
\end{equation}

We can write an equation for the evolution of the \(c_n(t)\): 
%
\begin{equation}
  i \hbar \dot{c_n} (t) = \sum_m V_{nm} \exp(i \omega_{nm} t) c_m(t) 
\,,
\end{equation}
%
where \(\omega_{nm} = (E_n - E_m) / \hbar\) and \(V_{nm} = \bra{n} V \ket{m} \). This is a matrix equation for the coefficient vector. 

\paragraph{Time-dep perturbations}

If we define the interaction-picture evolution operator as \(\ket{\alpha, t}  = U_I (t) \ket{\alpha, 0} \) we have its evolution as \(i \hbar \partial_t U_I = V_I U_I\).

For small times \(U_I \approx \mathbb{1}\), so we can integrate the Schrödinger equation: 
%
\begin{subequations}
    \begin{align}
        U_I &= \mathbb{1} + \frac{1}{i \hbar} \int_0^t V_I(t') U_I(t') \dd{t'}  \\
        &= \mathbb{1} + \frac{1}{i \hbar}\int_0^t V_I(t')\qty(
            \mathbb{1} + \frac{1}{i \hbar} \int_0^{t'} V_I(t'') U_I(t'') \dd{t''}
        ) \dd{t'}  \\
        &= \mathbb{1} + \frac{1}{i \hbar}\int_0^t V_I(t') \dd{t'} + \frac{1}{(i \hbar)^2} \int_0^t \int_0^{t'}V_I(t')V_I(t'') \dd{t'} \dd{t''} + o(V_I^2)
\end{align}
\,.
\end{subequations}
%

Now, if we start on a base ket \(\ket{i} \), the evolution coefficients \(c_n(t)\) will be given by the matrix elements \(\bra{n} U_I (t) \ket{i} \).
We can compute these to any order in \(V_I\), by taking the components of the previous equation and applying the following computation any time we have the components of \(V_I\): 
%
\begin{equation}
  \bra{n} V_I \ket{i} = \bra{n} U_0 ^\dag V U_0 \ket{i} = \exp(i \omega_{ni} t) V_{ni} 
\,,
\end{equation}
%
since the \(\ket{n} \) are eigenstates of the unperturbed Hamiltonian. 

\subsection{The propagator}

If \(H \ket{\alpha '} = \alpha ' \ket{\alpha '} \), then the evolution operator can be decomposed as 
%
\begin{equation}
  U (t) = \sum _{\alpha '} \exp(\frac{E_{\alpha '} t}{i \hbar}) \dyad{\alpha '}  
\,.
\end{equation}

This can be written in the position basis as a Green function by contracting with two position vectors: 
%
\begin{equation}
  \bra{x'} U(t) \ket{x''} =
  \sum _{\alpha '} \exp(\frac{E_{\alpha '} t}{i \hbar}) \braket{x'}{\alpha '} \braket{\alpha '}{x''} 
  \overset{\text{def}}{=}  K(x', x''; t)
\,,
\end{equation}
%
and with this we can directly compute the evolution at a generic time: \(\psi (x'', t) =  \int \dd[3]{x'} K(x'', x'; t) \psi (x')\). It is effectively the transition amplitude: \(K = \braket{x'', t}{x', 0} \) when seen in the Heisenberg picture (since we are evolving a base ket).

\begin{enumerate}
    \item \(K(x', x'', t)\) satisfies the Schrödinger equation, since it is a sum of terms which do;
    \item \(\lim_{t \rightarrow 0} K(x', x'', t) = \delta^3 (x', x'')\).
\end{enumerate}

\subsection{Some useful propagators}

\paragraph{Free particle}

We consider \(H = p^2/2m\); the momentum eigenstates are \(p \ket{p'} = p' \ket{p'} \), and they are also energy eigenstates with \(H \ket{p'}  = ((p')^2/2m) \ket{p'} \).

We compute: 
%
\begin{equation}
  K(x', x'', t) = \int \dd{p'} \braket{x''}{p'} \braket{p'}{x'} \exp(\frac{(p')^2 t}{i \hbar 2m }) 
\,,
\end{equation}
%
and recall that \(\braket{x}{p} = \exp(-px/ i \hbar) / \sqrt{2 \pi \hbar}  \). We simplify the exponent to get a Gaussian integral: it is known that
%
\begin{equation}
  \int _{\mathbb{R}} \dd{x} \exp(- i \alpha x^2) = \sqrt{\frac{\pi}{i \alpha }} 
\,,
\end{equation}
%
therefore in the end we get: 
%
\begin{equation}
  K(x', x'', t) = \frac{1}{2 \pi \hbar} \exp(\frac{im (x'' - x')^2}{2 \hbar t}) \sqrt{\frac{2 m \pi \hbar}{it}} 
\,,
\end{equation}
%
which for \(t \rightarrow 0\) is in the form
\(\exp({(x' - x'')}^2/t) \sqrt{t} \rightarrow \delta (x''- x') \).

\paragraph{Harmonic oscillator}

We consider \(H = p^2/2m + m \omega^2 x^2 / 2\). It is known that the eigenfunctions are given by the Hermite polynomials: 
%
\begin{equation}
  \braket{x'}{n} = \frac{1}{\pi^{1/4} \sqrt{2^{n} n!}} \frac{1}{x_0^{1/2}} H_n y \exp(-\frac{(x/x_0)^2}{2}) 
\,,
\end{equation}
%
where \(x_0 = \sqrt{\hbar / (m \omega )} \) (both masses and frequencies are inverse lengths in natural units!). We also know the eigenenergies, \(E_n = \hbar \omega (n+1/2)\). We can then compute away, to finally get: 
%
\begin{equation}
  K = \sqrt{\frac{m \omega }{2 i \pi \hbar \sin(\omega t))}}
  \exp(\frac{im \omega \qty(((x'')^2+(x')^2)\cos(\omega t)-2x'x'' )}{2 \hbar \sin(\omega t) })  
\,.
\end{equation}
%

\section{The Path Integral}

We can time-slice the interval between a certain time \(0 = t_0\) and another time \(t = t_N\) in \(N\) parts. Then, evolving the system with a the propagator for each one, we get: 
%
\begin{equation}
  K(x_N, x_0, t) = 
  \int \qty(\prod _{i=1}^{N-1} \dd[]{x_i}) \qty(\prod_{i=0}^{N-1}\braket{x_{i+1}, t_{i+1}}{x_i, t_i})  
\,.
\end{equation}

We call the time-slice \(\epsilon = t/N\). We will expand the in \(\epsilon \) up to first order the evolution operator \(\exp(H \epsilon /i \hbar) \).

\end{document}
